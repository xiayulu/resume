% !TEX TS-program = xelatex
% !TEX encoding = UTF-8 Unicode
% !Mode:: "TeX:UTF-8"

\documentclass{resume}
\usepackage{zh_CN-Adobefonts_external} % Simplified Chinese Support using external fonts (./fonts/zh_CN-Adobe/)
% \usepackage{NotoSansSC_external}
% \usepackage{NotoSerifCJKsc_external}
% \usepackage{zh_CN-Adobefonts_internal} % Simplified Chinese Support using system fonts
\usepackage{linespacing_fix} % disable extra space before next section
\usepackage{cite}

\begin{document}
\pagenumbering{gobble} % suppress displaying page number

\name{周千森}

\basicInfo{
  \email{2414711236@qq.com} \textperiodcentered\ 
  \phone{(+86) 152-7190-0848} \textperiodcentered\ 
}

\section{\faGraduationCap\  教育背景}
\datedsubsection{\textbf{武汉大学}, 武汉}{2019年9月 -- 2023年6月}
\textit{本科}\ 计算机科学与技术, 2023 年 6 月毕业

\section{\faUsers\ 实习/项目经历}

\datedsubsection{\textbf{WHULambda} \quad 武汉大学}{2021年9月 -- 至今}
\role{全栈工程师}{复责整个网站的设计,实现与维护}
WHULambda(https://whu-lambda.com) 是一个在线学习与知识分享网站,使用技术与解决方案:
\begin{itemize}
  \item 前端框架: react.js
  \item 服务端渲染:Next.js
  \item css框架:tailwindcss
  \item UI 框架:AntD
  \item 后端框架:go-fiber
  \item 缓存服务:go-redis
  \item 数据库:gorm + postgresql
  \item 异步任务:asynq
  \item 配置框架:viper
  \item 图片存储:云对象存储
  \item 文章存储:git repo
  \item 文章协作:Git Smart HTTP
\end{itemize}

项目难点与挑战:\\
1. 前后端分离开发:前后端身份认证,RestfulAPI 设计,前后端接口联合调用。\\
2. 需求分析与流程设计:这个项目的初衷是做一个知识分享平台,需要不断发掘需求与功能。\\
3. 基础组件定制:目前有很多基础组件,例如 Web 编辑器, 是网上找的,但随着开发的深入,正在定制相关基础组件。\\
4. 网站运营:随着项目的深入,功能越来越多,业务越来越复杂,对自己的能力水平要求越来越高。

\datedsubsection{\textbf{知识分享项目} \quad 武汉大学}{2020年9月 -- 至今}
\role{Markdown}{个人负责的内容创作}
https://github.com/xiayulu/corn 是个人学习笔记的总结。我做这个项目的原因是:
\begin{itemize}
  \item 网络现存的学习资料局限于各种流行的技术,但基础知识方面的比较少
  \item 大学课堂学习效益较低, 做知识分享能加深自己的理解
  \item 期末复习
\end{itemize}

在做这个项目的时候,遇到的最大问题就是基础知识领域内的创作非常困难,越是基础的东西越难,我的解决方法是坚持创作。

\datedsubsection{\textbf{知识产权实习}\quad 武汉大学}{2022年6月 -- 至今}
\role{实习}{负责整理知识产权相关数据}
\begin{itemize}
  \item 知识产权平台调研
  \item 知识产权数据整理
  \item 知识产权政策调研
\end{itemize}

\datedsubsection{\textbf{编译原理课设}\quad 武汉大学}{2022年4月 -- 5月}
\role{课设}{负责实现 OCaml 部分功能}
\begin{itemize}
  \item 采用 Rust 语言实现
  \item 通过阅读 OCaml 语法规范精炼出 ZCaml 语法规则
  \item 通过递归下降实现了语法树解析
  \item 实现了简单的类型检查
\end{itemize}

% Reference Test
%\datedsubsection{\textbf{Paper Title\cite{zaharia2012resilient}}}{May. 2015}
%An xxx optimized for xxx\cite{verma2015large}
%\begin{itemize}
%  \item main contribution
%\end{itemize}

\section{\faCogs\ IT 技能}
% increase linespacing [parsep=0.5ex]
\begin{itemize}[parsep=0.5ex]
  \item 熟练掌握 HTML, CSS, JavaScript 等基础知识;
  \item 熟练使用 React, tailwindcss, bootstrap Web 框架;
  \item 熟悉 DevOps 开发流程;
  \item 熟练使用 Go 语言开发 Restful API;
\end{itemize}

\section{\faInfo\ 其他技能}
% increase linespacing [parsep=0.5ex]
\begin{itemize}[parsep=0.5ex]
  \item 语言: 英语
  \item 写作:Markdown, Latex
  \item 设计:PS, AI
  \item 知乎: https://www.zhihu.com/people/xiayulu
  \item GitHub: https://github.com/xiayulu
\end{itemize}

%% Reference
%\newpage
%\bibliographystyle{IEEETran}
%\bibliography{mycite}
\end{document}
